
\subsection{Expert Feedback}

We demonstrated the three use cases to three experts in illustration and animation in order to gather their feedback.
Each use case was presented in form a video of the respective animated transition.
We supplied the experts with questionnaires that contained specific questions for each use case.
We asked, whether the animated transition could adequately answer the questions posed in the introduction of Section \ref{sec:usecases}. Additionally, we asked them for general feedback to the transition and what specific information they were able to infer from it that they could not infer from the target representation alone. Finally, we asked how long it would take them to re-create such an animation manually in a 3D modeling tool such as Maya.



\subsubsection{Exploded Views}
%********could the animation answer the posed question?

%-g- think this animated transition using a layer-pealing exploded view approach does a good job to answer the posed question
%-iw- The presentation makes it clear what structures reside within others.  
%One possible improvement is to zoom in to smaller structures so that they are more visible.
%If the focus is on the virus and its compartments, there is too much focus on the blood plasma proteins, which really aren't part of the virus at all. These could be decreased or eliminated to give more focus on the virus itself.
%-p- Yes the transition does reveal the main structural components of the virus, and you do see how each part fits together. Aren't the blue molecules, the outermost molecules, water? If so, why did you include them if you are only interested in the virus.  

%pro:
%-layer-pealing exploded view approach does a good job to answer the posed question.
%-The presentation makes it clear what structures reside within others.
%%%%-does reveal the main structural components of the virus, and you do see how each part fits together.

%con:
%-One possible improvement is to zoom in to smaller structures so that they are more visible. If the focus is on the virus and its compartments, there is too much focus on the blood plasma proteins, which really aren't part of the virus at all. These could be decreased or eliminated to give more focus on the virus itself. 


%*****************
%********positiv:

%-g- I like the clamshell explosion technique- it does a nice job of revealing the spatial relationships without distorting or deleting any information. I think better lighting, e.g. global illumination/ambient occlusion, could show the dents left when large items are removed, for example the inverse hemisphere left in the blood plasma when the HIV is removed.
%-iw- Overall, I like the clarity of this "fracturing" type of transition.  I'm not sure, however, why there are molecules that do not move with the upper or lower sections, and float around in the middle.
%-p- I like that it's simple movement, smooth movement and good speed. I don't dislike anything.

%pro:
%-I like the clamshell explosion technique- it does a nice job of revealing the spatial relationships without distorting or deleting any information. 
%-I like the clarity of this "fracturing" type of transition.
%-I like that it's simple movement, smooth movement and good speed.

%con:
%- I think better lighting, e.g. global illumination/ambient occlusion, could show the dents left when large items are removed, for example the inverse hemisphere left in the blood plasma when the HIV is removed.
%-I'm not sure, however, why there are molecules that do not move with the upper or lower sections, and float around in the middle.

%*****************
%********advantage of transition:

%-g- I could infer where each piece had come from.  As mentioned above, if the lighting had effects like ambient occlusion or global illumination, we would be able to infer where towards the left each piece had come from.
%-iw- Simply looking at the final image would make it far less clear what the relationships between the different compartments are.
%-p- How each component fit into each other

%pro:
%-I could infer where each piece had come from.
%-Simply looking at the final image would make it far less clear what the relationships between the different compartments are.
%-How each component fit into each other


%con:
%-if the lighting had effects like ambient occlusion or global illumination, we would be able to infer where towards the left each piece had come from.


All experts confirmed that the transition does reveal the main structural components of the virus, as well as what structures reside within others. %how the individual parts fit together.
\textit{"I like the clamshell explosion technique - it does a nice job of revealing the spatial relationships without distorting or deleting any information."; "I like the clarity of this "fracturing" type of transition."}

One illustrator suggested to zoom in to smaller structures on the right side, so that they are more visible. This could be achieved by applying a layout metamorpher that translates the smaller structures closer to the camera. One illustrator was not sure about the meaning of the individual floating molecules that did not move with the upper and lower exploded sections. Another illustrator suggested better lighting in order to show the \textit{"dents left when large items are removed, for example the inverse hemisphere left in the blood plasma when the HIV."}

All illustrators confirmed that the transition was crucial for understanding the final image. \textit{"Simply looking at the final image would make it far less clear what the relationships between the different compartments are."}

%Regarding the final presentation of the transition (target state), 

%*****************
%********pro/con final img:

%pro:
%-I like how the matryoshka doll approach (clamshell exploded view), reveals the hierarchical of the various compartments of HIV and its environment.
%-Overall, it looks nice (color, rendering style)

%con:
%- When thinking about publication standards, one problem with the final image is that the space may not be used efficiently since things get smaller and smaller in height towards the right.
%-it would be nice if the smaller viral structures, which probably should be the focus of the animation, were bigger so that the detail could be more readily seen. 


%*****************
%********duration:
%-g- After loading the representation into C4D using cellPACK, it would take me about 30 minutes to setup the clamshell opening sequences and another 30 minutes to an hour to keyframe, rig, or script the more detailed pieces that travel with the extracted components as they are pulled from the models towards the right.  A model this size might take an additional 30-90 minutes to render.
%3h

%-iw- Assuming I had all of the structures/geometry for the model, it would probably take me a day or two to create this type of visualization and render it.
%2days

%-p- I've never tried to do this exact seperation/animation, but i don't think it would take too long. The virus structure loads into Maya in an organized way so that I don't really remember it being too difficult to manage.


The experts' estimated duration for the manual re-creation of the explosion transition ranged between three hours and two days. With our scripting interface, the respective metamorphers can be parametrized within within half an hour.




\subsubsection{Schematization}
%*****************
%********could the animation answer the posed question?

%-g- Feel free to pull any of the more detailed notes I emailed last month about this one:
%Yes, this transition does a great job of taking a densely packed model and simplifying it to reveal the composition of molecules that are isolated to each compartment, while still retaining a simplified rendition of the compartment (organic wavy colored compartment shapes.)


%-iw- Again, there is a lot of screen space here that's dedicated to the blood plasma and less for the virus.  Many of the proteins that are most presented most clearly are not part of the virus itself, but are outside of the virus, and many molecules that exist in the virus (such as integrase, envelope protein, other viral factors) are not shown at all or are not shown clearly.  The membrane surrounding the virus is also not clearly shown.

%-p- It does help show what the different types of molecules look like, because the animation takes away the layers that obstruct the deeper molecules. but it doesn't answer what types of molecules there are, labeling would actually answer that question. Maybe some people know all the different types of molecules by heart just by looking at them, but some don't. It gives no indication on quantity.

%pro:
%-"this transition does a great job of taking a densely packed model and simplifying it to reveal the composition of molecules that are isolated to each compartment, while still retaining a simplified rendition of the compartment (organic wavy colored compartment shapes.)"
%- does help show what the different types of molecules look like, because the animation takes away the layers that obstruct the deeper molecules.

%con:
%-lot of screen space here that's dedicated to the blood plasma and less for the virus.  Many of the proteins that are most presented most clearly are not part of the virus itself, but are outside of the virus, and many molecules that exist in the virus (such as integrase, envelope protein, other viral factors) are not shown at all or are not shown clearly.  The membrane surrounding the virus is also not clearly shown.
%-doesn't answer what types of molecules there are, labeling would actually answer that question.



%*****************
%********positiv:

%-g- Overall, I like the simplified representation and the brick of blue goo that still gives an impression of the shape and the relationship of the color scheme where molecular color schemes match the simpler fills that they become. This does a great job of still showing the compartmentalizations and comparment-to-compartment relationships.
%A major thing missing- the matrix of the HIV Is packed with proteins (this is the main thing our published models show, but in this transition, the HIV matrix (space between the envelope and the capsid) is completely empty and the proteins are missing.
%-iw- I like the painting style (Goodsell-like). 
%-p-  The transition itself is really nice, for the same reasons that I like the other one.

%pro:
%-like the simplified representation and the brick of blue goo that still gives an impression of the shape and the relationship of the color scheme where molecular color schemes match the simpler fills that they become. 
%This does a great job of still showing the compartmentalizations and comparment-to-compartment relationships.
%-like the painting style (Goodsell-like).

%con:
%- major thing missing- the matrix of the HIV Is packed with proteins (this is the main thing our published models show, but in this transition, the HIV matrix (space between the envelope and the capsid) is completely empty and the proteins are missing.



%*****************
%********advantage of trans:

%-g- In seeing the original model, I could truly appreciate how dense things were.  In the cleaned up final image, that density is harder to appreciate.
%-iw- The starting point gives an indication of crowdedness and relative abundance of protein, while the final image does not.
%-p- The transition let's the viewer see the complexity of the structure, where the final image is simplified.

%pro:
%-In seeing the original model, I could truly appreciate how dense things were.  In the cleaned up final image, that density is harder to appreciate.
%-he starting point gives an indication of crowdedness and relative abundance of protein, while the final image does not.
%-The transition let's the viewer see the complexity of the structure, where the final image is simplified.
%con:
%---


All experts confirmed that the animation successfully conveyed the compartmentalizations and compartment-to-compartment relationships in the virus. And also the "painting style" that the image space metamorpher achieved was complemented. The transition was stated to give a good impression \textit{"where molecular color schemes match the simpler fills that they become."}
One expert was particularly fond of this transition. \textit{"This transition does a great job of taking a densely packed model and simplifying it to reveal the composition of molecules that are isolated to each compartment, while still retaining a simplified rendition of the compartment (organic wavy colored compartment shapes.)"}

One expert was more interested in the inner parts of the virus and would have ignored the outer blood serum in her design. In our pipeline, this update is easily achievable, for instance, with a re-specification of the subsets that in the schema layout metamorpher processes. 

The experts appreciated that the transition conveyed additional information that would not have been visible with a simple depiction of the original and target state.
The transition \textit{"does help show what the different types of molecules look like, because the animation takes away the layers that obstruct the deeper molecules."
"The transition let's the viewer see the complexity of the structure, where the final image is simplified."}

%*****************
%********pro/con final img:

%-g- same
%-iw- The final representation only shows a small subset of viral proteins.
%-p- -----

%pro:
%---
%con:
%-The final representation only shows a small subset of viral proteins.

%*****************
%********duration:

%-g- This would likely take me between 4 and 6 hours to recreate with Cinema 4D.
%4h

%-iw- If I already had all of the geometry, I would guess that this type of visualization would take up to a week to create.
%1week

%-p- In Maya, I would do several sets of still renders by rendering the top layer, then erase the molecules I don't want to see anymore, then render the next layer. and then do the transition in post production in After Effects. The actual post production transition I would do with an animated mask reveal, it's very simple and would not take long. Setting up the Maya file would take the longest.

The effort for the manual re-creation of the transition to a schematic representation was deemed higher than for the explosion. Estimations lay between four hours and one week. A parametrization of the necessary metamorphers in our pipeline takes would take up approximately the same amount of time as the setup for the explosion.









\subsubsection{Histograms}
%*****************
%********could the animation answer the posed question?

%-g- I don’t thinks this current implementation gives me an easy to understand or accurate depiction of the compartment volumes. If I wanted to compare the volumes of the compartment, I would prefer to have a bar in the graph for each compartment.  In this version I have to mentally stack each molecular tower to sum them up to understand how tall the compartments are relative to each other.  The lighting on the bars is hard to read, so the 3rd dimension in the bar graph could be interepreted as a taller bar instead of one that goes back into 3D space.  The camera moves during the transition which makes it hard for me to understand how big the original sphere-shaped particle was before the layers got pealed away.  Lastly, the bars are out of the safe area of the movie, at the border of the screen which makes them hard to read.
%-iw- I thought this animation was very unclear.  It could be improved by labeling (why are there so many blue-ish columns, orange-ish columns, etc, and what protein does each column represent?).  
%-p- I don't think it answers the relative volume of each compartment because it looks like it's just stacking the numbers of each type of molecule. So I can tell that there are more light blue molecules then red molecules. Also by the end I can't remember where the red tiles come from or the blue tiles, and I have no idea what each other represents. 



%*****************
%********positiv:

%-g- See above for critiques. I do like the continuous connected flow from the 3D structure to the bar graphs to help maintain a visual connection.  It is a fun piece to watch, but scientifically, I’m not sure how useful it can be until the issues described above are addressed.
%-iw- It's not clear to me what purpose this animation would serve.  It might be more useful to think about proteins in quantity rather than volume (how many of each protein is there?)
%-p- The transition is really nice, it looks interesting, the movement is great!



%*****************
%********advantage of trani:

%-g- Everything… simply because there is no before state in the final image, only a collection of lumpy 3D bar graphs with no labels, so it would be tough to understand where they came from or what they refer to.  In that regard the animation is nice because it makes a connection from the complex structure to the volumetric analysis.

%-iw- Without labels, the final representation is almost impossible to interpret on its own.  With the transition, it’s a bit more clear what the representation is meant to show, but it is still very difficult to interpret.

%-p- Nothing, the transition just looked cool. Obviously the target image says nothing alone, apart that there is more light blue somethings then red somethings. So the target image can't stand alone. But the particular transition doesn't provide addition information except maybe that the blue somethings came from the virus structure, but I don't know where it came from exactly.



%*****************
%********pro/con final img:

%-g- See above… its out of context, needs labels or some relationship to the original , e.g. if the original came back up somehow at the end it would be a moore complete comparison.
%-iw- ---
%-p- nothing in particular, needs labels, don't know what I'm actually looking at.

The transition to histograms was the most critical regarded, in comparison to the prior two use cases.
The illustrators uniformly appreciated the engangingness of the transition and the fact that it established a visual connection to the target representation, \textit{"I like the continuous connected flow from the 3D structure to the bar graphs to help maintain a visual connection. It is a fun piece to watch!" "Would be tough to understand where they came from or what they refer to [without the transition]."}
On the critical side, all three illustrators would have added labels to the target representation, in order to make their context more understandable. \textit{"The target image can't stand alone."}
We discuss why annotations are not part of our pipeline in Section \ref{sec:discuss}.
All other complaints are easily addressable with existing metamorphers. One expert preferred to have a bar in the graph for each compartment. This is achievable by creating a by-compartment hierarchy in the data structuring stage, before executing the bar layouts on the individual subsets. Another expert would have preferred 2D bars. This is achievable with object space morphing metamorpher that we demonstrate in our supplemental video. One suggestion to establish context without labels, was to add the original representation at the end of the animation. This is actually supported by the data restructuring and timing stages. Another suggestion was to encode the quantities of proteins instead of their collective volume. This would be easily achievable with a re-parametrization of the object space metamorpher.

%pro:
%-like the continuous connected flow from the 3D structure to the bar graphs to help maintain a visual connection.  It is a fun piece to watch
%-The transition is really nice, it looks interesting, the movement is great!

%-would be tough to understand where they came from or what they refer to.
%-labels, the final representation is almost impossible to interpret on its own.  With the transition, it’s a bit more clear what the representation is meant to show, but it is still very difficult to interpret.
%-the target image can't stand alone.

%con:
%-no easy to understand or accurate depiction of the compartment volumes.
%-prefer to have a bar in the graph for each compartment. === doable with metamorphers
%-lighting on the bars is hard to read, so the 3rd dimension in the bar graph could be interpreted as a taller bar instead of one that goes back into 3D space. == 2d example
%- by the end I can't remember where the red tiles come from or the blue tiles, and I have no idea what each other represents.
%-It might be more useful to think about proteins in quantity rather than volume (how many of each protein is there?) == doable
%-target is out of context without labels
%-if the original came back up somehow at the end it would be a moore complete comparison. == doable
%-mentally stack each molecular tower to sum them up to understand how tall the compartments are relative to each other === also doable
%--navigation: The camera moves during the transition which makes it hard for me to understand how big the original sphere-shaped particle was before the layers got pealed away.  Lastly, the bars are out of the safe area of the movie, at the border of the screen which makes them hard to read.


%*****************
%********duration:

%-g- This would take me between 6 and 12 hours to create with rigging and a bit of scripting in Cinema 4D

%-iw- A week or less, provided I already had the geometry.

%-p- no one in their right mind would try to do that manually. If I needed to make a graph that showed the relative amount of each type of molecule (which I have never been asked to do) then I would try to find a clever way to find out how many molecules of each type there are, I would start with the original "recipe", usually there is a recipe and that would say how many molecules are of each type. I would then make a the graph manually. Again, I've never tried, but I would never try to do a transition like that manually.


The experts estimated the effort for the manual re-creation of this use case between six hours and a week. One expert even stated that she would not dare to create such a transition manually. In our scripting interface, this was actually the easiest transition to set-up, as the bar layout does not require much parametrization. 
