
\subsection{Expert Feedback}

We demonstrated the three use cases to three experts in illustration and animation in order to gather their feedback.
Each use case was presented in form a video of the respective animated transition.
We supplied the experts with questionnaires that contained specific questions for each use case.
We asked, whether the animated transition could adequately answer the questions posed in the introduction of Section \ref{sec:usecases}. Additionally, we asked them for general feedback to the transition and what specific information they were able to infer from it that they could not infer from the target representation alone. Finally, we asked how long it would take them to re-create such an animation manually in a 3D modeling tool such as Maya.



\subsubsection{Exploded Views}
%********could the animation answer the posed question?

%-g- think this animated transition using a layer-pealing exploded view approach does a good job to answer the posed question
%-iw- The presentation makes it clear what structures reside within others.  
%One possible improvement is to zoom in to smaller structures so that they are more visible.
%If the focus is on the virus and its compartments, there is too much focus on the blood plasma proteins, which really aren't part of the virus at all. These could be decreased or eliminated to give more focus on the virus itself.
%-p- Yes the transition does reveal the main structural components of the virus, and you do see how each part fits together. Aren't the blue molecules, the outermost molecules, water? If so, why did you include them if you are only interested in the virus.  

%pro:
%-layer-pealing exploded view approach does a good job to answer the posed question.
%-The presentation makes it clear what structures reside within others.
%%%%-does reveal the main structural components of the virus, and you do see how each part fits together.

%con:
%-One possible improvement is to zoom in to smaller structures so that they are more visible. If the focus is on the virus and its compartments, there is too much focus on the blood plasma proteins, which really aren't part of the virus at all. These could be decreased or eliminated to give more focus on the virus itself. 


%*****************
%********positiv:

%-g- I like the clamshell explosion technique- it does a nice job of revealing the spatial relationships without distorting or deleting any information. I think better lighting, e.g. global illumination/ambient occlusion, could show the dents left when large items are removed, for example the inverse hemisphere left in the blood plasma when the HIV is removed.
%-iw- Overall, I like the clarity of this "fracturing" type of transition.  I'm not sure, however, why there are molecules that do not move with the upper or lower sections, and float around in the middle.
%-p- I like that it's simple movement, smooth movement and good speed. I don't dislike anything.

%pro:
%-I like the clamshell explosion technique- it does a nice job of revealing the spatial relationships without distorting or deleting any information. 
%-I like the clarity of this "fracturing" type of transition.
%-I like that it's simple movement, smooth movement and good speed.

%con:
%- I think better lighting, e.g. global illumination/ambient occlusion, could show the dents left when large items are removed, for example the inverse hemisphere left in the blood plasma when the HIV is removed.
%-I'm not sure, however, why there are molecules that do not move with the upper or lower sections, and float around in the middle.






%*****************
%********advantage of transition:

%-g- I could infer where each piece had come from.  As mentioned above, if the lighting had effects like ambient occlusion or global illumination, we would be able to infer where towards the left each piece had come from.
%-iw- Simply looking at the final image would make it far less clear what the relationships between the different compartments are.
%-p- How each component fit into each other

%pro:
%-I could infer where each piece had come from.
%-Simply looking at the final image would make it far less clear what the relationships between the different compartments are.
%-How each component fit into each other


%con:
%-if the lighting had effects like ambient occlusion or global illumination, we would be able to infer where towards the left each piece had come from.


All experts confirmed that the transition does reveal the main structural components of the virus, as well as what structures reside within others. %how the individual parts fit together.
\textit{"I like the clamshell explosion technique - it does a nice job of revealing the spatial relationships without distorting or deleting any information."; "I like the clarity of this "fracturing" type of transition."}

One illustrator suggested to zoom in to smaller structures on the right side, so that they are more visible. This could be achieved by applying a layout metamorpher that translates the smaller structures closer to the camera. One illustrator was not sure about the meaning of the individual floating molecules that did not move with the upper and lower exploded sections. Another illustrator suggested better lighting in order to show the \textit{"dents left when large items are removed, for example the inverse hemisphere left in the blood plasma when the HIV."}

All illustrators confirmed that the transition was crucial for understanding the final image. \textit{"Simply looking at the final image would make it far less clear what the relationships between the different compartments are."}

%Regarding the final presentation of the transition (target state), 

%*****************
%********pro/con final img:

%pro:
%-I like how the matryoshka doll approach (clamshell exploded view), reveals the hierarchical �nesting� of the various compartments of HIV and its environment.
%-Overall, it looks nice (color, rendering style)

%con:
%- When thinking about publication standards, one problem with the final image is that the space may not be used efficiently since things get smaller and smaller in height towards the right.
%-it would be nice if the smaller viral structures, which probably should be the focus of the animation, were bigger so that the detail could be more readily seen. 


%*****************
%********duration:
%-g- After loading the representation into C4D using cellPACK, it would take me about 30 minutes to setup the clamshell opening sequences and another 30 minutes to an hour to keyframe, rig, or script the more detailed pieces that travel with the extracted components as they are pulled from the models towards the right.  A model this size might take an additional 30-90 minutes to render.
%3h

%-iw- Assuming I had all of the structures/geometry for the model, it would probably take me a day or two to create this type of visualization and render it.
%2days

%-p- I've never tried to do this exact seperation/animation, but i don't think it would take too long. The virus structure loads into Maya in an organized way so that I don't really remember it being too difficult to manage.


The experts' estimated duration for the manual re-creation of the explosion transition ranged between three hours and two days. With our scripting interface, the respective metamorphers can be parametrized within a few minutes.
















\subsubsection{Schematization}
could the animation answer the posed question?

positiv:

advantage to static final img:

pro/con final img:

duration:










\subsubsection{Histograms}
could the animation answer the posed question?

positiv:

advantage to static final img:

pro/con final img:

duration:



